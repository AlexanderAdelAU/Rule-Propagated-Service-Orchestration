\documentclass{article}
\usepackage{amsmath,amssymb,amsthm}
\usepackage{algorithm}
\usepackage{algorithmic}

\title{Enhanced Buffered Token Synchronization Nets (BTSN) Mathematical Model}
\author{Mathematical Framework with Synchronization ID Semantics}
\date{\today}

\begin{document}

\maketitle

\section{Core Mathematical Model}

\subsection{Enhanced Definition}

A \textbf{Buffered Token Synchronization Net} is defined as a 9-tuple:

\begin{equation}
N = (P, T_{in}, T_{out}, F, \tau, \beta, \sigma, \theta, \rho)
\end{equation}

where:
\begin{itemize}
\item $P$ = finite set of places (service execution nodes)
\item $T_{in}$ = finite set of input transitions (synchronization coordinators)
\item $T_{out}$ = finite set of output transitions (distribution coordinators)
\item $F \subseteq (T_{out} \times T_{in}) \cup (T_{in} \times P) \cup (P \times T_{out})$ = flow relation
\item $\tau: T_{in} \rightarrow 2^{T_{out}}$ = synchronization requirements
\item $\beta: (T_{in} \cup T_{out}) \rightarrow \mathbb{N}$ = buffer capacity function
\item $\sigma: P \rightarrow (Token \rightarrow Token)$ = service execution function
\item $\theta: (T_{in} \cup T_{out}) \rightarrow \mathbb{N}$ = timeout function
\item $\rho: T_{in} \rightarrow SchedulingPolicy$ = scheduling policy function
\end{itemize}

\subsection{Enhanced Token Model with Correlation Semantics}

A token is defined as a 7-tuple:
\begin{equation}
tok = (id, correlationId, tokenNumber, data, timestamp, pathHistory, notAfter)
\end{equation}

where:
\begin{itemize}
\item $id \in TokenId$ = unique token identifier
\item $correlationId \in CorrelationId$ = synchronization group identifier
\item $tokenNumber \in \mathbb{N}$ = sequence number within workflow
\item $data$ = payload data
\item $timestamp \in \mathbb{R}^+$ = creation timestamp
\item $pathHistory \in T_{out}^*$ = sequence of transitions visited
\item $notAfter \in \mathbb{R}^+$ = expiration timestamp
\end{itemize}

\section{Token Identity and Correlation Management}

\subsection{Correlation ID Extraction Function}

The correlation ID extraction function maps any token ID to its synchronization group:
\begin{equation}
extractCorrelationId: TokenId \rightarrow CorrelationId
\end{equation}

For token IDs following the pattern \texttt{BASE-NNN-FFF}:
\begin{equation}
extractCorrelationId(id) = \begin{cases}
id & \text{if } id \text{ contains no fork suffix} \\
id[0:\text{lastIndexOf}("-")] & \text{if } id \text{ contains fork suffix}
\end{cases}
\end{equation}

\textbf{Example:} $extractCorrelationId(\text{"TOK-001-101"}) = \text{"TOK-001"}$

\subsection{Fork Generation Semantics}

For a base token with correlation ID $c$ and token number $n$, fork generation follows:
\begin{equation}
forkId(c, n, i) = c \parallel "-" \parallel (n \times 100 + i)
\end{equation}

where $i \in \{1, 2, ..., k\}$ for $k$ target transitions, and $\parallel$ denotes string concatenation.

\textbf{Example:} Base token \texttt{"TOK-001"} with $tokenNumber = 1$ generates:
\begin{itemize}
\item Fork 1: \texttt{"TOK-001-101"}
\item Fork 2: \texttt{"TOK-001-102"}
\item Fork 3: \texttt{"TOK-001-103"}
\end{itemize}

\subsection{Token State Transitions}

The fundamental token transformation patterns are:
\begin{align}
BaseToken &\xrightarrow{fork} \{ForkToken_1, ForkToken_2, ..., ForkToken_k\} \\
\{ForkToken_1, ..., ForkToken_k\} &\xrightarrow{join} BaseToken \\
ForkToken_i &\xrightarrow{no-join} BaseToken
\end{align}

\section{Synchronization Semantics}

\subsection{Synchronization Group Formation}

For input transition $t_{in}$ and correlation ID $c$:
\begin{equation}
SyncGroup(t_{in}, c) = \{tok \in Buffer(t_{in}) : extractCorrelationId(tok.id) = c\}
\end{equation}

\subsection{Source Tracking Function}

The immediate source of a token is determined by its path history:
\begin{equation}
source(tok) = \begin{cases}
\text{"START"} & \text{if } |tok.pathHistory| = 0 \\
\text{last}(tok.pathHistory) & \text{otherwise}
\end{cases}
\end{equation}

\subsection{Synchronization Readiness Predicate}

An input transition $t_{in}$ can synchronize tokens with correlation ID $c$ when:
\begin{equation}
canSync(t_{in}, c) \iff \forall s \in \tau(t_{in}) : \exists tok \in SyncGroup(t_{in}, c) : source(tok) = s \land \neg expired(tok)
\end{equation}

\section{Join vs No-Join Transition Semantics}

\subsection{Transition Classification}

Input transitions are classified based on their synchronization requirements:
\begin{equation}
TransitionType(t_{in}) = \begin{cases}
\text{NoJoin} & \text{if } |\tau(t_{in})| \leq 1 \\
\text{JoinRequired} & \text{if } |\tau(t_{in})| > 1
\end{cases}
\end{equation}

\subsection{No-Join Conversion}

For no-join transitions, fork tokens are immediately converted to base tokens:
\begin{equation}
convertToBase(forkToken) = baseToken
\end{equation}
where:
\begin{itemize}
\item $baseToken.id = forkToken.correlationId$
\item $baseToken.correlationId = forkToken.correlationId$
\item All other fields preserved
\end{itemize}

\subsection{Join Synchronization}

For join-required transitions, multiple fork tokens are combined:
\begin{equation}
join(\{tok_1, tok_2, ..., tok_k\}) = joinedToken
\end{equation}
where:
\begin{itemize}
\item $joinedToken.id = \text{common correlation ID}$
\item $joinedToken.data = \bigcup_{i=1}^k tok_i.data$
\item $joinedToken.timestamp = \min_{i=1}^k tok_i.timestamp$
\end{itemize}

\section{Scheduling and Fairness}

\subsection{Priority-Based Scheduling}

The scheduling policy function $\rho$ defines token processing order:
\begin{equation}
\rho(t_{in}) : CorrelationId^* \rightarrow CorrelationId^*
\end{equation}

\textbf{Lowest-ID-First Policy:}
\begin{equation}
\rho_{LIF}(correlationIds) = \text{sort}(correlationIds, \text{lexicographic})
\end{equation}

\subsection{Fairness Guarantee}

Under the lowest-ID-first policy, starvation is prevented by ensuring that for any correlation ID $c$:
\begin{equation}
\forall c' < c : \text{if } canSync(t_{in}, c') \text{ then } c' \text{ processes before } c
\end{equation}

\section{Buffer Management Strategy}

\subsection{Adaptive Buffer Sizing}

Buffer capacity is determined by synchronization complexity:
\begin{equation}
\beta_{adaptive}(t_{in}) = \begin{cases}
150 & \text{if } |\tau(t_{in})| \geq 3 \text{ (complex synchronization)} \\
50 & \text{if } |\tau(t_{in})| = 2 \text{ (pair synchronization)} \\
20 & \text{if } |\tau(t_{in})| \leq 1 \text{ (no synchronization)}
\end{cases}
\end{equation}

\subsection{Buffer Overflow Handling}

When $|Buffer(t_{in})| = \beta(t_{in})$:
\begin{equation}
handleOverflow(t_{in}, newToken) = \begin{cases}
\text{discard}(newToken) & \text{if FIFO policy} \\
\text{discard}(\text{oldest expired}) & \text{if cleanup possible} \\
\text{reject}(newToken) & \text{otherwise}
\end{cases}
\end{equation}

\section{Temporal Semantics}

\subsection{Wait Time Calculation}

For correlation ID $c$ at transition $t_{in}$:
\begin{itemize}
\item $firstArrival(t_{in}, c)$ = timestamp of first token in correlation group
\item $syncCompletion(t_{in}, c)$ = timestamp when synchronization completes
\end{itemize}

\begin{equation}
waitTime(t_{in}, c) = syncCompletion(t_{in}, c) - firstArrival(t_{in}, c)
\end{equation}

\subsection{Processing vs Wait Time Distinction}

\begin{itemize}
\item \textbf{Processing Time}: Time spent in place $p$ executing service function $\sigma(p)$
\item \textbf{Wait Time}: Time spent in $Buffer(t_{in})$ awaiting synchronization
\end{itemize}

\begin{equation}
totalLatency(tok) = \sum_{t_{in} \in path(tok)} waitTime(t_{in}) + \sum_{p \in path(tok)} processingTime(p)
\end{equation}

\section{Formal Firing Rules}

\subsection{Input Transition Firing Rule}

An input transition $t_{in}$ fires for correlation group $c$ when:
\begin{equation}
Fire(t_{in}, c) \iff canSync(t_{in}, c) \land outputPlace(t_{in}).isAvailable() \land c = \min(\text{readyCorrelations}(t_{in}))
\end{equation}

\subsection{Output Transition Firing Rule}

An output transition $t_{out}$ fires when:
\begin{equation}
Fire(t_{out}) \iff \exists tok \in Buffer(t_{out}) : \neg expired(tok)
\end{equation}

\section{Correctness Properties}

\subsection{Token Conservation with Fork/Join}

\begin{theorem}[Token Conservation]
For any correlation ID $c$, the number of active tokens satisfies:
\begin{equation}
\sum_{t \in T_{in} \cup T_{out}} |SyncGroup(t, c)| + |ServiceTokens(c)| = \text{constant modulo completion/timeout}
\end{equation}
\end{theorem}

\subsection{Synchronization Completeness}

\begin{theorem}[Synchronization Completeness]
Under proper buffer sizing and timeout configuration:
\begin{equation}
\forall c \in CorrelationId : \forall t_{in} \in T_{in} : canSync(t_{in}, c) \Rightarrow \Diamond Fire(t_{in}, c)
\end{equation}
\end{theorem}

\subsection{Starvation Freedom}

\begin{theorem}[Starvation Freedom]
Under lowest-ID-first scheduling:
\begin{equation}
\forall c \in CorrelationId : \Diamond (\text{complete}(c) \lor \text{timeout}(c))
\end{equation}
\end{theorem}

\section{Implementation Mapping}

\subsection{Buffer Data Structure}

\begin{equation}
Buffer(t_{in}) = \{(tok, arrivalTime) : tok \in Token \land arrivalTime \in \mathbb{R}^+\}
\end{equation}

\subsection{Correlation Tracking}

\begin{equation}
CorrelationMap(t_{in}) : CorrelationId \rightarrow \{sourceId \rightarrow timestamp\}
\end{equation}

\subsection{Synchronization Check Algorithm}

\begin{algorithm}
\caption{Synchronization Check}
\begin{algorithmic}
\REQUIRE Input transition $t_{in}$
\STATE $correlationGroups \leftarrow groupBy(Buffer(t_{in}), extractCorrelationId)$
\STATE $sortedIds \leftarrow sort(correlationGroups.keys(), lexicographic)$
\FOR{each $c$ in $sortedIds$}
    \IF{$canSync(t_{in}, c)$ AND $outputPlace(t_{in}).isAvailable()$}
        \STATE $tokens \leftarrow correlationGroups[c]$
        \STATE $resultToken \leftarrow processTokens(t_{in}, c, tokens)$
        \STATE $outputPlace(t_{in}).accept(resultToken)$
        \RETURN \COMMENT{Process only one group per cycle}
    \ENDIF
\ENDFOR
\end{algorithmic}
\end{algorithm}

\section{Architectural Constraints (Preserved)}

The enhanced model maintains all original BTSN constraints:

\begin{align}
\forall p \in P: |^{\bullet}p| &= 1 \land |p^{\bullet}| = 1 \tag{service isolation}\\
\forall t_{in} \in T_{in}: |t_{in}^{\bullet}| &= 1 \tag{synchronizer output constraint}\\
\forall t_{out} \in T_{out}: |^{\bullet}t_{out}| &= 1 \tag{distributor input constraint}
\end{align}

\section{Complexity Analysis}

\begin{theorem}[Computational Complexity]
For a BTSN with enhanced synchronization semantics:
\begin{itemize}
\item Fork Generation: $O(k)$ where $k$ = number of target transitions
\item Synchronization Check: $O(|\tau(t_{in})|)$ per correlation group
\item Scheduling: $O(n \log n)$ where $n$ = number of ready correlation groups
\item Buffer Management: $O(1)$ for insertion, $O(m)$ for cleanup where $m$ = buffer size
\item Correlation Tracking: $O(1)$ per token arrival
\end{itemize}
\end{theorem}

\section{Conclusion}

This enhanced mathematical model provides the formal foundation for sophisticated synchronization ID management and fork/join semantics. The model now mathematically captures:

\begin{itemize}
\item Correlation-based token identity and grouping
\item Fork ID generation and management patterns
\item Join vs no-join transition semantics
\item Priority-based scheduling with fairness guarantees
\item Adaptive buffer management strategies
\item Comprehensive temporal analysis of wait vs processing times
\end{itemize}

The model serves as the theoretical foundation for implementing correct and efficient BTSN systems with complex synchronization requirements.

\end{document}